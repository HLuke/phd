\documentclass[authoryear,review,3p]{elsarticle}
\usepackage[usenames,dvipsnames]{xcolor}
\usepackage[normalem]{ulem}

\usepackage{lineno,hyperref}
\usepackage{amsmath,amsfonts,amssymb}

\usepackage{listings}
\usepackage{natbib}
\usepackage{graphicx}
\usepackage{algorithm}
\usepackage{algorithmic}
\usepackage{multirow}
\usepackage{subfigure}
\usepackage{hyperref}
\usepackage[acronym,nomain]{glossaries}

\begin{document}

\begin{frontmatter}

\title{Statistical Learning of Biological Structure in the Human Brain}

\author{Dr. med. Danilo Bzdok}

\end{frontmatter}

\bigskip
\bigskip
\bigskip
\centerline{
"But above all, master technique and produce original data; 
all the rest will follow."}
\centerline{Santiago Ram\'{o}n y Cajal}

\bigskip
\bigskip
\bigskip

\textbf{Supervisors\\}

Prof. Dr. rer.-nat. Stefan Conrad, Natural Science Faculty, HHU D\"usseldorf, Germany\\

Prof. Dr. med. Simon Eickhoff, Medical Faculty, HHU D\"usseldorf, Germany\\

Dr. Bertrand Thirion, INRIA, Saclay, France\\





\bigskip

\newpage
\section*{Publications related to the present dissertation}

\linebreak
Cumulative Impact Factor: $\approx$60

\linebreak
\textit{Original papers}

\textbf{Bzdok D}, Grisel O, Eickenberg M, Thirion B, Varoquaux G.
Semi-supervised Factored Logistic Regression for High-Dimensional
Neuroimaging Data. Under review at NIPS.

\textbf{Bzdok D}, Grisel O, Eickenberg M, Varoquaux G, Poupon C, Thirion B.
Network-network architecture: Generative models of task activity patterns.
Under review at Cerebral Cortex.

Bludau S*, \textbf{Bzdok D*}, Gruber O,
Kohn N, Riedl V, Müller V, Hoffstaedter F, Eickhoff SB.
Medial prefrontal aberrations in major depressive disorder
revealed by cytoarchitonically informed voxel-based morphometry.
\textit{American Journal of Psychiatry}, in press. *equal contributions

\textbf{Bzdok D*}, Hartwigsen G*, Reid A, Eickhoff SB.
A hierarchy of the left inferior parietal lobe in social cognition and
language.
\textit{Neuroscience and Biobehavioral Reviews}, in press. *equal contributions

\textbf{Bzdok D}, Heeger A, Langner R, Laird A, Fox P, Palomero-Gallagher,
Vogt BA, Zilles K, Eickhoff SB.
Subspecialization in the human posterior medial cortex.
\textit{Neuroimage}, in press.

Eickhoff SB, Laird AR, Fox PT, \textbf{Bzdok D}*, Hensel L*.
Functional segregation of the human dorsomedial prefrontal cortex.
\textit{Cerebral Cortex}, in press. *equal contributions

\textbf{Bzdok D}, Langner R, Schilbach L, Laird AR, Fox PT, Zilles K, Eickhoff SB.
Characterization of the temporo-parietal junction by combining data-driven
parcellation, complementary connectivity analyses, and functional decoding.
\textit{Neuroimage}, 2013.

\bigskip
\textit{Review and opinion papers}

Eickhoff SB, Thirion B, Varoquaux G, \textbf{Bzdok D}.
Connectivity-based parcellation: critique \& implications.
\textit{Human Brain Mapping}, in press.

Eickhoff, SB \& \textbf{Bzdok D}.
Neuroimaging and modeling. Where is the road to clinical application?
\textit{Der Psychiater}, 2014, in press. 

Eickhoff SB \& \textbf{Bzdok D}.
[Statistical meta-analyses in imaging neuroscience.]
\textit{Klinische Neurophysiologie}, 2013, 44:199-203.

\bigskip
\textit{Book chapters}

\linebreak
\textbf{Bzdok D} \& Eickhoff SB.
Statistical learning of the neurobiological of schizophrenia.
In: \textit{The neurobiology of schizophrenia}, Springer, Heidelberg.

\newpage

\section*{1 Introduction}

\subsection*{1.1 Analytical and heuristic accesses to nature}

The world around us is complex and volatile.
%
A large proportion of human research efforts are undertaken
in an \textit{analytical} fashion
based on 
"the unreasonable effectiveness of mathematics in the natural sciences."
This was phrased by the Hungarian-American
physicist, mathematician, and Nobel laureate Eugene P. Wigner (1960).
The language of mathematics is a powerful tool to
describe, formalize, and predict phenomena in nature.
Yet, the author emphasizes that the existence of natural regularities may not be imperative. He goes on to
say that it might be even more surprising that humans can actually
find these regularities and use them to their advantage.
Similarly, Albert Einstein said:
"The most incomprehensible, is that the world is comprehensible." 
%
Starting from human-conceived axiomes
we have always derived more complicated
properties of and relationships between mathematical objects by formal proofs
(Connes A., "A view of mathematics").
A logical pyramid of theoreoms is built that lead to always
more general asserations.
We also have detailed knowledge of the limitations of these mathematical
assertions.
%
On the one hand,
an identical regularity can often be equally well described in very distant
branches of mathematics.
On the other hand,
identical mathematical conclusions have reemerged from derivation of
a priori unrelated assertions.
%
Indeed, the same formal language has proofed very apt in
the study of completely unrelated topics and diverging scientific disciplines;
from the movements of celestial objects in the universe
studied in astronomy
to
the metabolism pathways governing the inner life of the cell
studied in biochemistry.
%
Many rules about the world can thus be perfectly grasped
(Hardy, "Apology").
As another example,
Fibonacci numbers (1, 1, 2, 3, 5, 8, 13, etc.)
reappear in many natural phenomena.
The number of petals of a flower and the spirals of a pinapple tend
to be Fibonacci sequences.
The family tree of honey bees is also governed by Fibonacci regularities.
Even the proportions of human finger bones follow this formalism.
Knowledge of such mathematical regularities
allows to impose logical structure on the external world.
%Ein sehr spannender und Fantasie anregender Start! 
%Aber hier ist streng genommen ein Wiederspruch 
%zu Einsteins überraschter Aussage, dass die Welt "comprehensive" ist. 
%Wenn dem wirklich so ist, dann muss man sie nicht "in ein Model zwingen" 
%oder "ihr Logik aufzwingen". Man könnte auch (positiver) 
%formulieren "knowledge of such mathematical regularities 
%opens our mind for the logical structure of the external world.
%Aber das diskutierst du ja auch im Folgenden
It remains an unresolved philosophical debate whether we have
\textit{discovered} or \textit{invented} mathematics.
Yet, there is probably no doubt
that mathematical conceptualization
evolves as a feature of human cultural evolution (Tomasello, 2001).
Even the most abstract mathematical concepts can
be exchanged between individuals. Consequently,
this knowledge resource can be easily passed
on across generations and geographical distances.
One may note that
there is usually consensus among mathematicians
about the architecture of their discipline.
From an anthropological perspective,
mathematical formalism
appears to be one of the most
powerful tools and most defining properties of the human species
(S. Dehaene, "The Number Sense").
Indeed,
Eugene Wigner concludes his praise of equations with the following words:
"The miracle of the appropriateness of the language of mathematics
for the formulation of the laws of physics is a wonderful gift
which we neither understand nor deserve. We should be grateful for
it and hope that it will remain valid in future research [...]."
(1960, p. 14).


However, this dogma has repeatedly been challenged formally
and empirically.
%
In formal approaches
mathematics were shown incomplete and inherently contradictory because,
in any axiomatic system, some true assertions cannot be proven
("incompleteness theorems", G\"odel, 1931).
Additionally,
it is possible to define a real number with equidistributed
digits that can however not be computed
("Chaitin's constant $\Omega$", Chaitin, 2006).
%
In empirical approaches
the omnipotence of mathematical equations has also been challenged
as the
best possible way to describe and predict nature
(Halevy et al., 2009; Hinton/LeCun/Bengio, 2015; Pietsch, 201?).
%
This shift in scientific discourse is made explicit by
three recent empirical observations:
\begin{enumerate}
  \item Sophisticated, more accurate models can be outperformed by
  simple models that are fit with increasing amounts of training data.
  \item Simple, inaccurate models trained with rich data can outperform
  models that have been designed according to
  extensive domain expertise.
  \item There appears to be a minimal threshold of training data
  such that the derived models suddenly exhibit emergence properties.
\end{enumerate}

One prominent example for modelling nature in such a \textit{heuristic} fashion is
automatic language translation of human text and speech.
Translation systems based on 
human-made grammar rules have perhaps never achieved
satisfactory success (P. Norvig, 2011, "On Chomsky").
That is, analytical approaches based on
thousands of book pages archiving domain expertise in form of
deterministic rules appeared insufficient for building
language models that can cope with real-world settings.
%
Statistical machine translation was more successful
by implementing probabilistic hidden Markov models (HMM)
as a heuristic approach.
%
The next word or N-gram is predicted
only by the one (order 1) or few (order n) preceding ones
with equal transition probabilities (Bengio, 199X "Markovian Models").
This special case of a recurrent neural network
computes the conditional probabilities of the next language element
depending on the most recent history of these elements
based on a dictionary of known elements.
The transition matrix of human language can be easily \textit{learned}
from data by observing a preferably long stream of
real-world language (i.e., a "corpus").
%
The class of HMMs
thus became a dominanting feature
of computational linguistics.
%
Today, virtually all professional translation software solutions
for both written and spoken language
are enabled by heuristic statistical models.


As an example from human biology,
we currently have only few means to predict the
toxicity of environmental chemicals and
potential effects of new drug compounds on health.
%
The complex and unknown phenomenon in nature here pertains
to the causal link from
a protein's \textit{known} primary structure
(i.e., 1D chain of amino acids)
to
its \textit{unknown} tertiary/quaternary structure
(i.e., the combination of 3D foldings
of the amino acid chain that subserves function).
%
Among the millions of existing proteins
we only know the tertiary/quaternary structure of
approximately 30.000 ones.
The structural configuration is however necessary to identify the position of
bindings sites for protein-protein interaction.
Knowledge of these sites in 3D space is crucially important
to infer that protein's interplay with the human body.
%
In this case,
the learning problem is to
derive a computational model from
massive pairs of known primary protein structure and
known protein properties, including toxicity,
by intentionally treating the 3D structure
as a black box.
%
State-of-the-art neural network algorithms
have very recently solved this heuristic learning problem
better than probably any previous approach in academia or industry
(Dahl et al., 2014; Unterthiner et al., 2015).
These investigators thus showed that biological activity
can be reliably predicted from single amino acid chains
even without recourse to biological domain expertise.
% "solved" klingt schon sehr final - man kann ja wahrscheinlich noch nicht
%reliabel die Toxizität vorhersagen?!
There might currently not exist
an analytical counterpart to such
structure-function mapping of proteins.
%
%
Similar probabilistic scenarios would include predicting
political outcomes,
optimizing advertisement strategies,
algorithmic trading in stock markets, and
controlling self-driving cars.
%Zitate würden diese teils auch dystopischen Visionen
%anderen Autoren zuordnen und dich damit entlasten

In sum,
humans can create machines to derive
algorithmic predictions from the data.
Observing a phenomenon in nature a sufficient number of times
might be sufficient to
algorithmically extract
the heuristics of its interaction behavior with the world.
This has recently entailed a shift
of attention
from model complexity to data complexity
and
from purely mathematical treatment to giving up some
human control to self-emergening patterns.
%Dem letzten Teil des Satzes würde ich evtl einen neuen Satz geben
In the absence of an analytical access,
simple statistical models can thus automatically
formalize diverse classes of natural phenomena
depending on the quality and quantity of
the available data resources.
%Schöne Zusammenfassung. Wichtiger Punkt, der für einige 
%selbstverständlich ist, aber für die Kritiker hier fehlt:
%Warum sollte der Mensch seine Kontrolle aufgeben und
%"self-emerging patterns" die Vorfahrt lassen? 


\subsection*{1.2 Two cultures of statistical modelling}

Statistics is a branch of mathematics that has arguably been the
overall most successful information science.
Statistics aims at extracting information from data
about the mechanisms in nature that generated these data.
%
Given its eclectic character, it may come as no surprise that statistics
has developed both analytical and heuristic strategies
to model regularities of phenomena in nature.
Yet, analytical and heuristic statistical cultures
have developed independently (Breiman, 2001).
They differ with regard to historical origin, mathematical foundation,
and modelling goal.
%Bin mir bei der Konnotation von "eclectic" nicht ganz sicher.
%In der Architektur gilt Eklektik mittlerweile manchmal auch als 
%zu "durcheinander" und ist in manchen Kreisen eher "out". 
%Ob Statistiker dem Begriff regelmäßig über den Weg laufen weiß ich nicht ;-)

The overwhelming majority of statisticians
follow an analytical regime by
adhering to \textit{classical statistics} (CS) for
\textit{data modelling}.
They hold that the phenomenon under study can be viewed as a black box
whose inner workings can be described by a small set of
underlying variables.
It is up to the statistician in charge
to choose the model that best reflects nature.
Data are then used to estimate the parameters of that pre-specified model.
Classical statistics has dominated research at the universities
for almost 90 years now.
%
Well known members of the CS family include for instance
Student's t-test, ANOVA,
and Chi-squared test.
\textit{Statistical hypothesis testing} has been introduced in the beginning
of the last century (Fisher, 1925; Neyman and Pearson, 1928).
The same approach is still practiced today in its original form (Goodman, 1999).  
%
The ensuing \textit{p-value} measures how likely it is
to observe the data at hand
assuming the non-preferred null hypothesis ($H_0$)
to find indirect evidence
for the preferred alternative hypothesis ($H_1$).
%
Despite the prevailing presence of p-values,
it has not been conceived by Fisher as an acid test
to judge existing versus non-existing effects in nature.
Rather, the intention was a preliminary tool to
filter which potential effects should be more explicitly tested (Nuzzo, 2014).
%
Notably, the drawn conclusions may be wrong
if the hand-selected model is a bad description of
the natural phenomenon under study.
%
Nevertheless, statistical hypothesis testing probably fit perfectly
in its time of inception and adoption.
In fact, it was designed for use with mechanical calculators
(Efron and Tibshirani, 1991).
Gaussian distributional assumptions
have been very useful in many instances to reach
mathematical convenience and, hence, computational tractability.
Additionally, it suited perfectly the Popperian view of
critical empiricism in academic discourse
(Popper, 1935/2005 "Logik der Forschung"):
scientific progress is to be made by continuous replacement of current
hypotheses by always more explanatory hypotheses
by means of \textit{verification} and \textit{falsification}.
The rationale behind hypothesis falsification
is that even a lot of evidence cannot confirm
a given theory in an \textit{inductive} way, 
while a single counter example is able to proof a theory wrong
in a \textit{deductive} way.
%
In sum,
classical statistics was mostly fashioned
for problems with few data points that can be grasped 
by plausible models with a small number of parameters chosen by the
investigator.


In contrast, only a small minority of statisticians
follow a heuristic regime by
adhering to \textit{statistical learning} (SL) for
\textit{algorithmic modelling}.
This statistical framework is frequently adopted by computer scientists,
physicists, engineers, and others without formal statistical brackground
that are typically working in industry
rather than academia (cf. Daniel and Wood, 1971).
%
They hold that natural phenomena
can be studied by estimating regularities in the inputs and
outputs to the black box without making assumptions
about its internal "true" mechanisms.
A statistical model is thus derived that expresses
relationships between the input and output variables
whose parameters are learned by training data (Abu-Mostafa, 2012).
Put differently, a new function with potentially thousands of
parameters is created
that can predict the output from the input alone,
without explicit programming model.
The input data thus need to represent different variants of
all relevant configurations of the examined phenomenon in nature.
Well-known members of
the SL family include for instance k-means clustering,
Lasso/Ridge regression, and support vector machine classification.
%
Please note that SL here summarizes the seemingly more specific terms
"data-mining", "pattern recognition", "artificial intelligence",
and "machine learning" that are often employed inconsistently.
The independent historical origin of CS and SL families is even
witnessed by the most basic terminology. 
In the CS literature inputs to statistical modeling
are traditionally called \textit{independent variables}
or, more recently, \textit{predictors},
while these are commonly referred to as \textit{features}
in the SL literature (Hastie et al., 2011).
%
When evaluating whether a certain problem is a possible target for SL
three requirements come into play (Abu-Mostafa, 2012):
\begin{enumerate}
  \item A regularity exists
(if there is no pattern, then it might still be worth trying SL).
  \item The regularity cannot be formalized analytically
(otherwise one can still apply SL, but it might not create the best model).
  \item We have data on the problem (the more, the better).
\end{enumerate}
%
This regime led to a surge of new
computer-intensive statistical techniques since 1980
that can be difficult to compute on a normal calculator
and
that are less concerned with mathematical tractability (Efron, 1991).
% In the 95'ies suppor vector machines (SVMs)
% proofed to be better than the back then de-facto-standard neural networks
% in both classification and regression problems (Valpnik, 1996).
This development has been flanked by changing properties of datasets that
are always higher-dimensional (i.e., more features per observation)
and
based
on larger samples (i.e., more observations).
This is a trend that is not specific to
neuroimaging research but also takes places
in other scientific disciplines,
% Bisher redest du ja noch sehr global über die Anwendung von CS und SL 
% Hier erwähnst du etwas plötzlich neuroimaging research
which affect everyday life by means of weather forecasts and economic predictions
(Manyika et al., 2011).
In sum,
statistical learning was mostly fashioned
for problems with many data points with largely unknown
data generating processes
that are emulated by a mathematical function
created en passant by a machine.


Importantly, some statistical
methods cannot be easily categorized by the CS-SL distinction.
Statistical methods do, in fact, span a continuum between the two poles of CS and SL
(Jordan/Frontiers in Massive Data, p. 61).
Nevertheless, the two families of statistical methods
can be easily distinguished by a number of archetypical properties.
Bayesian statistics are however orthogonal to the CS-SL distinction
and can be adopted in both methodological families in various flavors.
%
Neither can the terms univariate versus multivariate
(i.e., relying on one versus more than one input variable)
be clearly grouped into either CS or SL.
%
More generally,
neither CS nor SL can generally be considered superior.
This is captured by the \textit{no free lunch theorem}
stating that no single statistical strategy can#
consistently do better in all circumstances (Wolpert, 1996).
The challenge relies in choosing
the statistical approach that is best suited
to the neurobiological phenomenon under study and the neuroscientific research object at hand.


Regarding modelling goals, CS and SL exhibit various differences.
CS typically aims at modeling the black box by making a set of
accurate assumptions about its content,
e.g. the type of signal distribution.
Contrarily, SL typically aims at finding any way to model
the output of the black box from its input
while making the least assumptions possible (Abu-Mostafa et al., 2012).
In CS the phenomenon is therefore treated as partly known
(i.e., the stochastic processes that generated the data),
whereas in SL the phenomenon is treated as complex,
completely unknown, and partly unknowable.
It is in this way that CS tends to be
analytical
(i.e., imposing mathematical rigor on the phenomenon),
whereas SL tends to be
heuristic
(i.e., finding useful approximations to the phenomenon).
CS assumes a given statistical model at the beginning of the investigation,
whereas in SL the model is
generated in the process of the statistical investigation.
In more formal terms,
CS therefore closely relates to parametric statistics
for \textit{confirmatory} data analysis,
whereas SL closely relates to non-parametric statistics
for \textit{exploratory} data analysis
(Tukey, 1977 "Exploratory data analysis").
In more practical terms, CS is typically applied to experimental data
that were generated
the investigator controlled the variables of interest 
(i.e., the system under studied is perturbed),
while SL is typically applied to observational
data without such structured influence by the investigator
(i.e., the system is left unperturbed) (Domingos, 2012).
The work unit for CS is the quantified
significance associated with a statistical
relationship between few variables given a pre-specified model.
The work unit for SL is the quantified robustness of patterns
between many variables or, more generally,
the robustness of \textit{special structure} in the data (Hastie et al., 2011).
CS therefore tests for a particular structure in the data,
whereas SL explores and discovers structure in the data.
Formally, CS implements data modeling by
imposing an a priori model in a top-down manner,
whereas SL implements algorithmic modeling by fitting
a model as a function of the data at hand in a bottom-up manner.
%
Intuitively, the "truth" is believed
to be in the model (cf. Wigner, 1960) in a CS-constrained world,
while it is believed to be in the data
(cf. Halevy et al., 2009) in a SL-constrained world.


As a drastically oversimplified, yet useful, conclusion,
CS preassumes and tests \textit{a model for the data},
whereas SL learns \textit{a model from the data}.
%
Indeed,
both human and computer learning are theoretically
more conceivable in a probabilistic rather than
deterministic sense
(Abu-Mostafa et al., 2012; Dayan et al., 1995; Friston, 2010; Gregory, 1980).
Moreover, each probabilistc model can be viewed as a superclass
of a deterministic model (P. Norvig, "On Chomsky").
%
Taken together,
CS assumes that the data behave according to known mechanisms,
whereas SL exploits
computer algorithms to avoid the a-priori
specifications of data mechanism.
%würde exploit eher sachlicher formulieren - notfalls einfach „uses“



\subsection*{1.3 The human brain as a complex phenomenon in nature}
The human brain is a prime example of
a black box that is complex, mysterious, and perhaps in part unknowable.
It is frequently proposed that
the human brain might be the most complex object in the known universe
(Nature editorial, october 2014).
% schön eingeleitet, aber krass, wie anthropozentrisch
% selbst Forscher 2014 noch sind, oder? :-)
With the language from above,
the human brain might constitute a phenomon in nature that
can perhaps \textit{not} be perfectly grasped by mathematical formalism alone.
More concretely,
the \textit{most pertinent structure}
that we should assume for the human brain,
when measured by
contemporary functional neuroimaging techniques (cf. next passages),
is currently unknown.
Hence, the neuroimaging access to neuroscience can readily be framed as
a problem of \textit{representation learning} (Bengio, 2014).
It is conceivable
that this task can be solved without exhaustive
neurobiological micro-/meso-/macro-level knowledge (Bostrom, 2014).
This is always more supported by empirical evidence
(e.g., Helmstaedter, M. et al. "Connectomic reconstruction" 2013 Nature)
and
it is a contention that is embraced by the present dissertation.


From a global perspective,
the molecular difference between our genetic equipment and that
of our closest ancestors, the non-human primate, turns out to be 
strikingly small.
This has encouraged the conviction that one or very few key genetic
adaptations in the primate lineage have unchained an avalanche
of cognitive and cultural inventions that led up to today's civilization
(Tomasello, 2001).
That is, the human species might be much more defined by the
increasingly fast cultural evolution rather the ramifications
of slow biological evolution.
Crucial cognitive improvements,
such as the emergence of verbal language, 
might have fueled cultural improvements that, in turn, enabled
further cognitive improvements and inventions et cetera pp.
This form of \textit{online learning} is a very plausible and decisive
property of intact tissue of the central nervous system.
%ist online learning ein üblicher Begriff?
%
As a first challenge in brain science,
it might therefore be impossible to cleanly dissect
the nature-nurture interplay into independent contributing factors that act
during
phylogeny (i.e., development of the species)
and
ontogeny (i.e., development of an individual organism).
%
In this sense,
investigating the limits between "nature" and "culture" in the human brain
might equate with asking a paradoxical question
(Dehaene \& Cohen, 2007).
%
Instead,
a necessary factor for the high level of abstraction in human culture
might have precisely been the inextricability, due to bidirectional influence, of
neurobiological plasticity and relentless cultural exchange
between human individuals
in a non-stop, autopoietic optimization process
(Vygotsky 1978, "Mind in Society"; Luhmann 1984, "Soziale Systeme";
Bengio 2013, "Evolving Culture").


Given this recent acceleration in cultural evolution
(cf. Paul Virilio, "Open Skype"),
it might be
rather unlikely that the human brain has developed dedicated
neuronal populations to subserve the panoply of novel behaviors.
Rather, evolutionarily recent mental skills
(e.g., reading and writing, explicit pedagogy, and
symbolic mathematics)
might be realized by recombining low-level circuits that initially
developed for other functional roles.
This view has become known as
"neural reuse" and "neural recycling" hypotheses
(Anderson, 2010; Dehaene \& Cohen, 2007).
Non-human primates are lacking many of the sophisticated
mental operations that
are crucially important for maintaining human societies
(Mesulam, 1998; Tomasello, 2003).
In fact,
the "social brain hypothesis" states that our
computationally powerful brains are not an adaptation to
solve problems posed by the physical environment,
but for successfully coping with increasingly complex human social systems
(Humphrey, 1984; Byrne et al., 1988; Dunbar and Shultz, 2007).
Yet, it is becoming increasingly clear that socialaffective processing
in the human brain is probably realized by domain-general
brain regions and networks not specific to maintaining social interactions
(Bzdok et al., 2015 "Neurobiology of Morality"; Behrens et al., 2009 "Computation";
Barret et al., 2013).
These considerations entail a second challenge in brain science:
It is probably impossible to know what purpose neural
processing in a given part of the brain has originally evolved to serve.
We can only observe external manifestations and correlative relationships of
this latent biological purpose.
%Der Abschnitt endet sehr gut, klingt am Anfang aber vielleicht noch etwas zu
%Evolutions-Märchen-haft - vielleicht kann man die Überleitung von Recycling
%zu Social brain Hypothese noch geradliniger machen? Weil eigentlich ist die
%social brain hyp. ja kein „fact“, sondern eine weitere hypothese


Importantly,
no two human brains are alike.
Quite the opposite,
they differ with regard to
the morphology of gyri and sulci,
the topology of cytoarchitectonically and chemoarchitectonically
distinguishable brain areas,
the axonal connections linking these brain areas,
as well as the history of their sensory inputs.
%
The extent of a brain area and its inter-individual variability
can be quantitatively examined with its relation to cognition and behavior,
that is,
performance in psychological tasks in the healthy or diseased brain.
For instance, the volume of the amygdala is linked to
interindividual differences in memory performance as well as
many other (temporally transient) states and
(temporally enduring) traits.
%
As third challenge in brain science,
it is currently unknown how interindividual differences
in behavioral facets are mediated on the brain-level.
%
The renowned neuroanatomist Santiago Ram\'{o}n y Cajal wrote (1909):
"The complexity of the nervous system is so great,
its various association systems and cell masses so numerous and
complex, and challenging,
that understanding will forever lie beyond our most committed efforts."
%
More specifically, it remains largely elusive
whether distinct behavioral differences
between individuals are associated with changes of
cell bodies, dendrites, axonal connections, and/or glial cells
(Kanai et al., 2011).
That is,
we do not have clear understanding of how
this set of microstructures interact to
solve neural computation problems,
let alone their interindividual differences.
%
From a methodological perspective,
volumetric modelling techniques
conventionally employed in the neuroimaging field
are na\"ive to
many types of possible morphological differences.
For instance,
it is currently difficult to statistically grasp
inversely proportional left and right hemisphere volumes
or
a medical condition that randomly
affects either the left or the right brain
per individual
(Ashburner et al., 2011).


Worth to be proposed as an independent challenge of
brain science, the secret of interhemispheric
asymmetry is yet to be unveiled.
The connectivity differences between the left and right brain are
for instance currently underresearched.
They are even hardly known in the monkey (Stephan, 2007)
that usually serves a fallback system for human
connectivity investigations (Mesulam, 2012 "The evolving").
In humans, the majority of homologous brain areas feature
direct anatomical connections.
Nevertheless, as two textbook examples,
why the language and attention processes typically lateralize to
the left and right hemisphere, respectively,
is currently understood only in modest fragments
(Corbetta 2000; Stephan et al., 2003; Price et al., 2010).


It is further unlikely that we
will reach exhaustive understanding of
the human brain by mere
\textit{observational}, as opposed to \textit{interventional},
classes of research methods
(cf. J. Pearl, 2000 "Causality").
This idea is reflected in
Edward O. Wilson's words
"disturb Nature and see if she reveals a secret"
as well as in
G. M. Shepherd's words
"Nothing in neuroscience makes sense except in the light of behavior."
%
Purposely induced focal lesions of brain tissue in rats have early
been systemically related to resulting differences in
behavioral performance indices
(Franz and Lashley, 1917).
In hamsters, cats, and monkeys,
decortication entails only small sensory or motor effects,
while such tissue impairments of the neocortex in humans
result in much more pronounced and less reversible 
functional deficits
(Lashley, 1952; MacLean, 1982),
which points to increasing corticalization of brain function.
In humans, brain lesion studies have been
the most common approaches
to localize brain functions
until about 20 years ago.
%
However, inferring neurobiological insight from lesion findings constitutes
yet another challenge to brain science.
It constitutes an overly simplistic conclusion that
changes in behavior after destroying brain tissue in a
circumscribed brain area directly reveals functional roles
of that brain area (Young, 2000).
It is a limitation of these studies that
they attempt to derive the \textit{normal} function of
an area from the effects of \textit{damage} to that area.
%
First,
the destroyed brain area might  primarily subserve inhibitory effects,
such that abolition can increase neural processing subserved in remote areas
mediated by network connections.
Second,
a large fraction of human lesion cases are stroke patients.
The spectrum of lesion patterns found in these populations
is however seriously limited by the existing spectrum of
brain vessel anatomy
(e.g., the majority of ischemic strokes affect the Arteria cerebri media).
Third,
there is probably not a single psychiatric disorder that would be
characterized by very \textit{focal} (as opposed to distributed)
differences in brain structure
(cf. Goodkind, 2015 JAMA).
%
More generally, it is still a matter of debate whether
structure (i.e., locally specific micro- and chemoarchitecture),
connectivity (i.e., short- and long-rang axonal targets),
and function (i.e., lesion-induced behavioral changes)
reflect three  viewpoints on the same heterogeneity of
a particular brain area
(Passingham et al., 2002; Kelly et al., 2012 Neuroimage).


Each area in the brain exhibits
activation patterns of neuronal populations with
oscillatory regularities.
These oscillatory circles and their associated behaviors are
highly preserved in mammalian evolution
(Buzsaki, 2013 "Scaling brain size").
%
Perhaps since Hubel and Wiesel's (1965) description of increasingly
complex processing of neurons in the primary visual cortex
neuroscientists tend to think information processing as
serial sequences of sensory bottom-up and
modulating top-down information streams.
Axonal feedforward and feedback connections
are indeed a very good predictor of \textit{what} the next processing step is.
Yet, brain oscillations are capable of predicting \textit{when}
this next processing step will occur.
%
Oscillation measured by EEG and MEG techniques might be the most
attractive access to another challenge in brain science:
\textit{the binding problem} (Singer, 1999; Engel, 2001; Varela, 2001).
We are far from understanding how
environmental perturbation by multi-sensory stimulation
is coherently integrated and linked with prior experience into
a holistic higher-order percept via
spatially distributed and temporally coherent electrophysiological activity.
%
In animals, oscillatory but not spiking activity of neuronal populations appears
to be closely associated with sensory input processing.
The interpretation of oscillation findings is however demanding.
This is because they
simultaneously reflect
a maintenance equilibrium, sensitivity to external stimuli, and
formation of processing outputs.
For instance,
perception of environmental stimuli
is an intrinsically probabilistic process with nonidentical
results depending on the state of ongoing oscillatory circles.
Additionally,
different "rhythms" (i.e., frequency bands) flank each other
in a same brain area in an interacting fashion.
The same rythm can reflect different categories of computational processes
in different brain areas and networks. Some brain structures are
characterized by specific rythms that may not be found in the rest
of the brain.
Different frequency bands can subserve a same cognitive process, while
different cognitive processes can be realized by the same frequency bands.
%
Finally,
high frequencies govern large-scale networks in the brain that, in turn,
influence small local neuronal spaces with slow oscillatory patterns.


Also from a philosophical perspective
the neuroscientist faces problems when articulating observations of
phenomena in the brain.
%
For instance, brain areas or experimental effects are frequently
described according to "emotional" versus "cognitive"
interpretational categories.
However, this class of judgments implicitly preassumes
the neurobiological validity of traditional psychological categories.
That is, it assumes that those two concepts have
a discrete representation in measurable neurobiology.
Yet, as another major challenge to brain science,
it remains elusive how and to what extent psychological terms,
such as "emotion" and "cognition" (Pessoa, 2008; Van Overwalle, 2011),
map onto regional brain responses
(Laird et al., 2009; Mesulam, 1998; Poldrack, 2006). 
Potentially unjustified a-priori hypotheses are imposed
on the organization of human brain systems.
%
It should hence be carefully called into question what terms
are an adequate word choice to refer to discrete
neurobiological processes.
More globally,
confusion introduced by human language itself is at the origin
of many scientific problems
(Wittgenstein, 1953/2001 "Philosophical Investigations").
The grammatical and lexical constraints of human language might be
too tight to allow for unequivocal description of the diverse
circumstances humans encounter in science and ever-day life.
According to Wittgenstein the meaning of language is primarily
defined by its practical use in concrete situations,
rather than decontextualized abstractions necessarily pre-shared
by interlocutors.
Words might not have an objective meaning
equally accessible to and understood by everybody
(e.g., also specialisation alters consciousness according to
Habermas, 1984/87 "Theorie des kommunikativen Handelns" volume 1/2).
This is all the more the case for language descriptions
of phenomena that do not occur in every-day reality.
In this sense,
discussing subtleties of abstract neurobiological concepts,
which can hardly be practically experienced,
are frequent subject to ambiguity, thus leading to
unnoticed misunderstanding and unresolvable paradoxes
(cf. Bostrom, 2002; Watzlawick et al., 1967).
Biological processes in the brain are an instance of
such not directly experienceable phenomena underdetermined
by human language that entail interpretative conundrum.
%
More concretely,
there is still no community-wide consensus on
a comprehensive description system of
human mental operations (Poldrack, 2006 and 2011).
This has caused considerable heterogeneity in
how neuroimaging experiments have been motivated and conducted.
Moreover, it resulted in frequently inconsistent findings
that are difficult to reconcile conceptually.
%
Statistically, rather than falsely rejecting (i.e., type I error) or
falsely accepting (i.e., type II error) the null hypothesis,
previous experimental fMRI studies motivated by
preassumed psychological categories might have commited
"the error of the third kind" (Kimball et al., 1957):
providing an accurate answer to an inadequate research question.
It might be more useful to
strive towards "an approximate answer to the right question"
(John W. Tukey) given that "all models are wrong" (George Box) anyways.
%
In sum,
cognitive neuroscience has so far heavily relied on concepts
historically inherited from traditional,
non-neurobiological scientific disciplines.
These considerations are especially relevant to investigations
whose conclusions heavily rely on CS.
Statistical hypothesis testing makes the
strong implicit assumption that
the semantic concepts used to formulate 
the null and alternative hypotheses are "true"
(i.e., neurobiologically congruent).
% In den letzten Sätzen wird es etwas sprunghaft zw den Themen:
% (Philosophie-Statistik-dann die SUmmary CS suggeriert (faelschlich) truth)

The last challenges to the neuroscientist
mentioned here are of epistemological origin.
Biology as a whole has a modest legacy in abstract theory.
This probably includes the history of the biology of the brain.
In particular,
the spectrum of permissible conclusions
that can be drawn from neuroscientific investigations is strongly conditioned
by the following three questions
(Carruthers, 2009; Dehaene, MBE 2007):
\begin{enumerate}
  \item Does the human brain offer sufficient computational resources
  to grasp, formalize, and predict itself?
  \item Is the human mind capable to reflect upon itself
by directly contemplating itself via introspection or by
indirectly contemplating an internalized self-model acquired through
interaction with others?
  \item To what extent is the self-reflexive description of the phenomenology of
  the human mind by the human mind itself immanently
  limited and paradoxical?
\end{enumerate}


Taken together,
there are many intricacies about neurobiology and 
the mosaic knowledge that we currently have about it.
%
Despite $\approx$200 years of neuroscience,
we are probably not even close to something like a unified theory of
brain function
that neuroscientists from different fields would accept
(cf. Friston, 2010 "Free energy principle"; Bar, 2009 "Predictions").
This caveat considerably complicates
the formulation of precise, neurobiologically
valid hypotheses that can be experimentally tested in targeted studies.
%
Therefore, it might be helpful to use heuristics-establishing
statistical approaches for pattern discovery instead of
classical statistics alone.
%
Discovering the mystery of the brain \textit{exclusively} by
successive falsification of
entirely human-conceived,
intimately language-dependent,
and dichotomically framed hypotheses
might be viewed as hubris by some
(cf. Cajal, 1909; Cohen, 1994).
%
Therefore,
the present dissertation is built on the assumption that
we might not reach an \textit{exhaustive analytical understanding}
of the brain any time soon
and that a more pragmatic access
may rely in the \textit{heuristic
approximation of brain mechanisms} by statistical learning models.
%
Such an attempt to learn patterns from data
would follow the same
direction as recent research developments in
language translation and drug discovery (cf. 1.1).



\subsection*{1.4 The curse of dimensionality}
Not only neurobiological and conceptual challenges,
but also the increasing quantities of analyzed data
put neuroscientific research to the test
(Gorgolewski \& Poldrack, NNR).
%
Today's neuroimaging methods offer very high resolution in
space (especially fMRI and PET)
and
time (especially EEG and MEG)
(Amunts et al., 2014 Science; Buzsaki \& Draguhn, 2009 Science).
%
The mere number of features poses serious
statistical challenges to the investigator.
It is the neuroscientific version of what Richard Bellman
called the \textit{curse of dimensionality} (1961).
%
At the root of the problem,
all data samples look virtually identical
in high-dimensional data scenarios.
%
Accustomed to regularities in 3D neighborhoods,
human intuition is often led astray in
how data behave in
input spaces with an extreme number of variables.



The more dimensions an input space spans,
the further the data points are away from each other
(Hastie et al., 2011).
Conter-intuitively,
measuring the distance between a randomly selected data point
and its closest uniformly distributed neighbors,
reveals a shell-like occurrence probability of
these neighbors, rather than a centered probability mass.
%
Put differently,
when approximating a hypersphere by a surrounding hypercube,
the probability mass of the hypercube
would almost entirely lie outside the hypersphere
(Domingos, 2012).
%
Put in yet another way,
a space divided into isotropic units grows exponentially in the
unit number with linearly increasing dimensionality.
As the main practical conclusion,
the amount of data necessary to populate these units
also grows exponentially
with linearly increasing input variables
(Bishop, PRML).
% Muss ich die letzten Sätze verstanden haben? !Critical info-density!



Additionally,
the target function is almost always unknown
in statistical learning investigations.
Hence, we frequently have no knowledge of whether or not
special structure may exist in the input data that can be exploited.
%
Knowledge of special structure of the phenomenon under study
can reduce both \textit{bias}
(i.e., difference between the target function and
the average of the function space derivable from a model)
and
\textit{variance}
(i.e., difference between the best approximating function 
from the function space and
the average of the function space).
This is a rare opportunity in SL because increasing,
for instance, the model complexity
typically increases the variance and lowers the bias, and vice versa.
%
In particular,
the problem of overfitting in SL has an immediate relationship
with the multiple-comparisons problem in CS
(Domingos, 2012).
%
The \textit{bias-variance decomposition} captures the fundamental
tradeoff in statistical modeling between
approximating the behavior of
the studied phenomenon and
generalizing to newly generated data describing that behavior. 



A peacefully coexisting conceptual framework exists in SL
that is independent
of the unknown target function.
The \textit{Vapnik-Chervonenkis (VC) dimensions}
formalize the circumstances
under which learning processes can be successful (Vapnik, 1989, 1996).
This comprises any instance of learning
from a number of observations
to derive heuristic rules that capture properties of phenomena in nature,
including learning in humans and machines.
Formally, the VC dimensions measure the
complexity capacity of a class of approximating functions
(i.e., the function space). 
%
Practically, good models have finite VC dimensions
and are therefore capable to generalize to new data.
Bad models have infinite VC dimensions that
are unable to make generalization conclusions on unseen data,
regardless of data quantity.



More concretely,
SL approaches that incorporate locally varying functions
in small \textit{isotropic} neighborhoods
will fail to generalize in high-dimensional data scenarios.
SL approaches that overcome the curse of dimensionality typically
incorporate an explicit or implicit metric for
\textit{anisotropic} neighborhoods
(Hastie et al., 2011).
%
It is the \textit{hyperparameters} that govern the
smoothing behavior of the imposed local neighborhoods.
%
In so doing,
the \textit{hypothesis set} (i.e., each function in the function space
represent a hypothetical solution to
the estimation problem) is hopefully reduced to
a reasonable pre-selection (cf. \textit{regularization}).
%
Guiding the statistical estimation process by
complexity restrictions can alleviate the curse of dimensionality.
First,
we can deliberately exclude members of the hypothesis set.
Viewed from the bias-variance trade-off, this calibrates
the sweet spot between underfitting and overfitting.
Viewed from Vapnik's statistical learning theory,
the VC dimensions can be reduced and thus the generalization performance
increased.
%
Second, there is an infinity of possiblities to restrict the hypothesis set.
Yet, these choices are typically guided by external knowledge beyond
the data at hand.
%
Third,
different complexity restrictions typically lead to different
best approximating functions.


In sum,
the choice of any statistical method constraints
the spectrum of possible results and of permissible interpretations.
Any scientific discovery in the brain is only valid in the
context of the complexity restrictions that have been imposed
on the neurobiological phenomenon of interest.
%
No single statistical strategy, be it SL, CS, or other,
can consistently
do better in all neuroscientific investigations
(Wolpert, 1996).
%
The present dissertation
is hence dedicated to the juggling with
complexity restrictions to
neurobiological reality as observed by fMRI scanning.



\subsection*{1.5 Imaging neuroscience}
Functional specialization in the Cortex cerebri of humans has been
investigated in the nineteenth
century predominantly by lesion reports
(Harlow, 1848, 1868; Broca, 1865;
Wernicke 1881 "Die acute, hämorrhagische Poliencephalitis superior").
Brain lesion studies and brain stimulation during surgery were the mainstay of
neuroscientific research for a long time,
until they were complemented by
axonal tracing studies for connectivity analysis in
animals (cf. Mesulam, 1976).
%
Today, functional magnetic resonance imaging (fMRI)
is the most frequently chosen approach for non-invasive,
in-vivo brain research in humans,
counting more than 1,000 new neuroimaging publications per year.
The impact of fMRI is explained by the availability
of brain scanners in medical institutions,
its non-invasivness,
and its significant spatial resolution (1-2 mm, Engel et al., 1997)
and temporal resolution (a few seconds, Jezzard 200X).
%
fMRI enables the localization of neural activity changes at the synapse by
means of measuring the accompanying changes in
the oxy-to-deoxyhaemoglobin ratio in local draining veins
(Roy et al., 1890; Ogawa et al. 1990/1993).
For instance, onset of vibratory stimulation of a participants' hand
entails regional accumulation of metabolic equivalents that cause
regional blood flow increase ("neurovascular coupling") in the
contralateral somatosensory cortex (Fox et al., 1986).
%
In particular,
the measured BOLD (blood oxygen-level dependent) signal exhibits
an initial dip after the onset of neural activity
increase that is attributed to the fast local increase
in deoxyhemoglobin.
The ensuing hyperperfusion and the
thus generated (relative) hyperoxygenation then dictate
the BOLD signal shape (i.e., "hemodynamic response function").
It is slightly different across the brain regions of an individual,
across individuals, and probably across different tasks.
Neural activation is finally followed by re-inhibition of
blood flow observable as an undershoot at the
end of the BOLD signal (Logothetis et al., 2001).
%
Juxtaposing neural activity and corresponding BOLD signals,
the BOLD signal is at least one order of magnitude noisier,
scales roughly linearly with neural activity, and
is better predicted by local field potentials than
multi-unit spiking activity. The BOLD signal is possibly more associated
with input to and processing in a local neuronal population
rather than its output. There is thus no clear-cut quantitative relation
between the spike rate of neuronal populations and the ensuing
BOLD response. Rather, the BOLD signal reflects a mixture of
transient spikes and continuous membrane potentials (Logothetis et al., 2004).
%
As a central property, there is a tradeoff between
coverage of sampled brain tissue, spatial and temporal resolution. 
For instance, augmenting the spatial resolution, while keeping
brain coverage constant, deteriorates the temporal resolution.
%
Finally,
the regional responses in single individuals are transformed
into a standard brain space
(i.e., "spatial normalization" into the
"Talairach-Tournoux" [cite] or
"Montreal Neurological Institute" [cite]
coordinate systems)
for comparability and statistical analysis on the group-level.


Based on local changes in cerebral blood flow,
experimental fMRI has provided
insight into the cerebral localization
of specific tasks related to sensory processing, motor actions,
and affective functions
(Brett et al., 2002).
This is achieved by
performing fMRI on an individual that lies
inside the scanner magnet
while attending and responding to psychological tasks,
compared to the absence of that task.
Usually, the neural correlates of a given task
(i.e., a mental process of interest) are isolated by subtraction
of the activation measured during a closely related task (i.e., control task)
that is supposed not to evoke the mental process of interest.
This relies on the principle of "pure insertion" that
cognitive subtraction between the psychological processes of both
target and control tasks is possible due to large absence of
interaction between them.
Although this assumption may not be tenable in many
practical cases,
the principle of direct task comparison has been widely adopted
since it has been shown to be
neurobiologically useful, as well as statistically robust and reproducible
(Friston, Zarahn, Josephs, Henson, & Dale, 1999).
%
In many instances,
analysis and interpretation of brain imaging data
is often performed by integrating additional
behavioral data (e.g., task reaction times in the simplest case).
Dozens of scans of a same experimental task
that cover metabolic changes in the whole brain are acquired
for enhanced sensitivity.
%
The spectrum of neuroimaging-compatible tasks is practically
only limited by the scanner surroundings
and the interdiction of head movements.
%
In this way, fMRI tasks have revealed the location patterns of
various regionally specific effects in health and disease.


In contrast,
in the absence of task (i.e., during mind wandering),
the human brain is not at rest.
While most fMRI studies focus on the minority of neural activity changes
conditioned by external stimulation,
increasing attention is devoted to
the majority of neural activity patterns
that underlie the biochemical maintenance of
the neural "house-keeping" architecture.
That is,
the BOLD signal can also be measured in a task-unconstrained fashion
by probing participants that lie in the scanner without following
a defined psychological task.
Participants are instructed to think of nothing in particular
let their minds go, and leave their eyes open/closed or
look at a fixation cross. During mind wandering humans typically mentally shift
between various heterogeneous types of thoughts, memories, and predictions.
This is why resting BOLD patterns are believed to reflect the repertoire of
cognitive operations that the human brain can perform (Smith et al., 2009).
%
From a neurophysiological perspective,
intra- and inter-neuronal activity continues
in the human brain's resting functionality.
The resting-state BOLD signal reflects fluctuations in
physiological signals recorded in the absence of task
as reflected in a voxels' time courses.
Importantly, the (small) amplitude of the resting-state signal is modulated by
transient psychological states (e.g., arousal, attention, and alertness),
but also cardiac and respiratory influences.
Indeed, the decomposability of this signal measurement into independent
components suggests a set of distinct influences rather
than one coherent signal pattern (Fukunaga et al., 2006).
More specifically, evidence exists in favor of a neuron-, metabolism-,
vasculature-, and oxygen-driven genesis of the resting-state BOLD signal. 
%
More specifically,
correlation analysis can detect temporal coincidence in
the spontaneous, slow fluctuations (~0.01 - 0.1 Hz) of rest BOLD.
This is taken as a measure of functional coordination between
topographically distant parts of the brain.
Measuring these coherent spatiotemporal couplings in resting-state BOLD
fluctuations yields a set of robust neural networks.
It led to the discovery of a set of so-called
\textit{resting-state networks}.
%
In sum, the biggest fraction of the various brain
signals does not correlate with a particular
behavior, stimulus, or experimental task.
These partially uncouple in a task setting,
but the relative change is small.
It is commonly agreed that the variability in the RS signal is
related to the individual's (unconstrained) mental operations.
It likely represents a physiological
instantiation of a human beings' default mental repertoire.


A property of the brain that we might not have
discovered without the advent of neuroimaging methods is the
so-called \textit{default mode network} (DMN).
The present dissertation is closely related
to this particular resting-state network
that is a pure result of serendipity
(Shulman et al., 1997; Gusnard et al., 2001).
%
15 years ago,
the DMN was initially proposed to be exclusive
in decreasing neural activity consistently during experimental
paradigms requiring stimulus-guided behavior.
That is, the DMN was believed to
increase neural activity in the idling, unconstrained mind and decrease activity
during stimulus-driven, goal-directed tasks (Gusnard et al., 2001).
On a macro-scale, the metabolic baseline turnover is not equally
distributed across the brain.
Interestingly, the brain areas of the DMN include the hot spots of
highest metabolic consumption that locate, first, to the
posterior cingulate cortex extending into the adjacent retrosplenial
cortex and precuneus and, second, to the medial prefrontal cortex
extending into the anterior cingulate cortex
(Raichle et al., 2001; Reivich et al., 1979).
It was later even argued that this network is systematically anti-correlated
with brain regions more active during task performance (Fox, et al., 2005).
Indeed, goal-directed task performance improves with increased activity
in saliency-related areas and decreased activity in default-mode
areas (Weissman et al., 2006). Conversely, increased activity in DMN areas
were linked to increased occurrence of
task-independent thoughts (i.e., mind-wandering) during task
execution (Mason et al., 2007). Two fMRI studies employing Granger causality
analysis further corroborated the anti-correlation by indicating negative
influence of the default-mode on the saliency
network (Pisapia et al., 2012) and vice versa (Sridharan et al., 2008).
This anti-correlation was recently challenged by repeated reports of brain regions
exhibiting both task-constrained and task-unconstrained increases in
neural activity (Buckner, et al., 2008). More specifically,
the DMN is now known to consistently increase neural activity
during a small set of complex cognitive tasks, including
the contemplation of others’ and one’s own mind states,
spatial navigation,
as well as scene construction processes
when envisioning past, fictitious, and future events
(Spreng, et al., 2009);
more generally, envisioning situations detached from reality.
%
It was speculated that the human brain might have evolved to,
by default, predict environmental events using mental imagery.
Constructing detached probabilistic scenes could thus influence perception
and behavior by estimating saliency and action outcomes.
This would invigorate a possible relationship between
the physiological baseline of the human brain
and an introspective psychological baseline
(Schilbach et al., 2008).
%
In sum, the DMN routinely defies neuroscientific intuitions
and challenges established methods.
Neuroimaging research on the DMN corroborated that this particular network
consistently decreases activity during externally focused mental tasks
and typically increases activity during a small
set of internally focused mental tasks.
It may reflect unfocused every-day mind wandering
in form of
continuous environmental tracking in a generative, integrative process.
Yet, we are not even close to certain knowledge of
what this might mean en detail.
% Das gilt für das DMN, aber auch für viele andere Netzwerke


\subsection*{1.6 Statistical learning approaches in brain imaging}

Everyday neuroimaging practice is still largely dominated by
analysis approaches drawn from classical statistics.
%
Much of the success of cognitive neuroscience
since the 1990'ies has been implemented in the
mass-univariate analysis of neuroimaging data
using the general linear model (GLM).
The GLM treats each volumetric pixel of brain scans (i.e., voxel)
as independent to perform
serial univariate statistics (Friston et al., 1995). 
%
Univariate approaches are recognized to be an excellent
test for topographical localization of neural activity,
i.e., a differential increase or decrease
of neural activity in individual brain voxels.



SL approaches promise to extend this
representational agenda of fMRI investigations
(i.e., analysis of activation localizations) 
to an informational agenda
(i.e., analysis of information patterns)
(Kriegeskorte et al., 2006; Mur et al., 2009).
%
SL approaches can elicit hidden quantities in neuroimaging data by
providing new pieces of evidence to four questions
(Brodersen, 2009; Pereira et al., 2009):
\begin{enumerate}
  \item\textit{Where} an information category is neurally processed?
  As SL techniques are inherently multivariate, the 
  coherent patterns of BOLD signal in voxel sets are localized.
  This extends the interpretational spectrum from mere
  increase/decrease of neural activity to
  the existence of complex combinations
  of distributed activity changes.
  \item\textit{Whether} a given information category
  is encoded by neural activity? This extends the interpretational
  spectrum to topographically similar but
  neurally distinct processes that potentially underlie
  different psychological concepts.
  \item\textit{When} an information category is generated, processed,
  and bound? When applying SL to BOLD time series, for instance,
  starting from
  experimental stimulus onset, the interpretational
  spectrum is extended to the evolution of predictive performance
  in the time dimension.
  \item\textit{How} an information category is neurally processed?
  The interpretational spectrum is extended to
  computational properties of the neural processes, including
  for instance, linearity versus nonlinearity as well as
  local versus distributed and isolated versus partially
  shared computational facets.
\end{enumerate}
More generally, multivariate information inference is typically
more potent than mass-univariate localization inference
because the latter is inherently focal and threshold-dependent
(Friston 2008).
%
The popularity and adoption of SL methods in neuroimaging
has steadily increased since
the attempt of "mind-reading" or "decoding" cognitive processes
from neural activity patterns
(Haynes and Rees, 2005; Kamitani and Tong, 2005).
The conceptual appeal has been complemented by
recent advances in computing power, memory resource, and
the increasing trend for creating large data repositories
(Poldrack and Gorgolewski, 2014).


More specially, GLM-based and SL-based analysis regimes in 
functional neuroimaging can be conceptualized as
complementary instances of
\textit{encoding models} and \textit{decoding models}
(Naselaris et al., 2008):

\begin{eqnarray}
  f\colon\ y_t \to \mathbf{X_t}\\
  g\colon\ \mathbf{X_t} \to y_t
\end{eqnarray}

where $(1)$ represents a (voxelwise) GLM as an encoding function and
$(2)$ for instance a (brainwise) linear classifier as a decoding function,
$X_t \in \mathbb{R}^{d}$ a 3D matrix of voxel values holding BOLD signals
in brain space,
$y_t \in \mathbb{N}^n$ a set of indicators of a psychological task or
mental context, and $t \in \mathbb{N}$ a time series of brain scans.
In a related vocabulary,
the encoding function is the basis for \textit{forward inference},
testing the probability of observing
neural activity in brain regions given knowledge of the psychological process
(Yarkoni et al., 2011).
The decoding function, in turn, is the basis for
\textit{backward inference},
testing the probability of a psychological process
being present given knowledge of neural activity in brain regions.
A main difference between encoding and decoding models pertains to
the direction of linearly mapping between brain space and feature space.
Nevertheless,
both encoding or decoding functions
can be viewed as a prediction task since,
for deciding on a relationship between an activity $X_t$ and a
context $y$ at time point $t$, the mapping direction is irrelevant.
Encoding models are superior to decoding models for
establishing which processing facets are preferentially
represented within brain regions.
Encoding models can also be easily compared to one another,
whereas
inference about brain represetations according to decoding models
reduces to model comparison.
An important advantage of decoding models is that
they lend themselves more naturally to examining the
correspondence between
brain activity in brain regions and indices of behavioral performance.
Decoding models are also more flexible than encoding models
in allowing
"identification" (Kay et al., 2008),
inferring a stimulus or task from a finite set based on brain activity,
and
"reconstruction" (Miyawaki et al., 2008; Thirion et al., 2006),
restoring a stimulus or task from brain activity.
%
In sum,
the classical functional neuroimaging localization
might be a weak choice to perform inference on
structure-function relationships
without formal modelling (Stephan, 2004),
whereas
decoding models are
readily applicable
for establishing complex structure between
high-dimensional neuroimaging data
and variables of interest.



Taken together,
as a complementary methodological family,
SL approaches are characterized by
a) making the least assumptions possible,
b) being more motivated by computational models rather than cognitive theory,
and c) automatically mining structured knowledge from data resources.
Even a
small-group fMRI study qualifies as a high-dimensional statistical problem.
A transition from parametric to non-parametric modeling
(e.g. Russell \& Norvig 2010, Ch. 18.8)
and
from data to algorithmic models (Breiman 2001)
has been prompted by an ongoing
drift towards more data-driven (i.e., fewer assumptions),
higher-dimensional (i.e., more features per observation) neuroimaging analyses
on larger samples of neuroimaging data (i.e., more observations).
%
Epiphenominal of the current clash between
CS and SL techniques in the neuroimaging field,
there is much
controversy about how they interact in everyday research practice
(e.g., whether or not cross-validated classification accuracies need to be
validated by p-values).
%
The crucial challenges in imaging neuroscience
might however lie in
\textit{mechanistic interpretability and understanding}
by way of generative, rather than descriminative, SL models.
(Brodersen et al., 2011).
An important step towards this goals is the question
whether
low-dimensional manifolds are
embedded within the high-dimensional neuroimaging data.
%Das Ende finde ich etwas überraschend. Die wichtige challenge,
%SL auch für mechanistic understanding zu nutzen, wird vorher ja noch recht
%wenig diskutiert, oder?


\section*{2 Unsupervised modelling of brain regions}

\subsection*{2.1 Motivation}
\subsection*{2.2 Methodological approach}
\subsection*{2.3 Experimental results}
\subsection*{2.4 Discussion}


\bigskip
\section*{3 Supervised modelling of brain networks}

\subsection*{3.1 Motivation}
\subsection*{3.2 Methodological approach}
\subsection*{3.3 Experimental results}
\subsection*{3.4 Discussion}

\bigskip
\section*{4 Semi-supervised modelling for structure discovery
and structure inference}

\subsection*{4.1 Motivation}
\subsection*{4.2 Methodological approach}
\subsection*{4.3 Experimental results}
\subsection*{4.4 Discussion}


\bigskip
\section*{5 Conclusion}



\bigskip
\section*{6 References}

\bibliographystyle{plainnat}
\bibliography{danilos_endnote_list}

\end{document}
