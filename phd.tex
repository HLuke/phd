\documentclass[authoryear,review,3p]{elsarticle}
\usepackage[usenames,dvipsnames]{xcolor}
\usepackage[normalem]{ulem}

\usepackage{lineno,hyperref}
\usepackage{amsmath,amsfonts,amssymb}

\usepackage{listings}
\usepackage{natbib}
\usepackage{graphicx}
\usepackage{algorithm}
\usepackage{algorithmic}
\usepackage{multirow}
\usepackage{subfigure}
\usepackage{hyperref}
\usepackage[acronym,nomain]{glossaries}

\begin{document}
  
\begin{frontmatter}

\title{Statistical Learning of Biological Structure in the Human Brain}

\author{Dr. med. Danilo Bzdok\\
Assistant professor of social and affective neuroscience}

\end{frontmatter}

%\center{
"'But above all, master technique and produce original data; 
all the rest will follow."'

Santiago Ram\'{o}n y Cajal
%}

\bigskip

\newpage
\section*{Publications related to the present dissertation}
\linebreak
\textit{Original papers}

\textbf{Bzdok D}, Grisel O, Eickenberg M, Thirion B, Varoquaux G.
Semi-supervised Factored Logistic Regression for High-Dimensional
Neuroimaging Data. Under review at NIPS.

\textbf{Bzdok D}, Grisel O, Eickenberg M, Varoquaux G, Poupon C, Thirion B.
Network-network architecture: Generative models of task activity patterns.
Under review at Cerebral Cortex.

Bludau S*, \textbf{Bzdok D*}, Gruber O,
Kohn N, Riedl V, Müller V, Hoffstaedter F, Eickhoff SB.
Medial prefrontal aberrations in major depressive disorder
revealed by cytoarchitonically informed voxel-based morphometry.
\textit{American Journal of Psychiatry}, in press. *equal contributions

\textbf{Bzdok D*}, Hartwigsen G*, Reid A, Eickhoff SB.
A hierarchy of the left inferior parietal lobe in social cognition and
language.
\textit{Neuroscience and Biobehavioral Reviews}, in press. *equal contributions

\textbf{Bzdok D}, Heeger A, Langner R, Laird A, Fox P, Palomero-Gallagher,
Vogt BA, Zilles K, Eickhoff SB.
Subspecialization in the human posterior medial cortex.
\textit{Neuroimage}, in press.

Eickhoff SB, Laird AR, Fox PT, \textbf{Bzdok D}*, Hensel L*.
Functional segregation of the human dorsomedial prefrontal cortex.
\textit{Cerebral Cortex}, in press. *equal contributions

\textbf{Bzdok D}, Langner R, Schilbach L, Laird AR, Fox PT, Zilles K, Eickhoff SB.
Characterization of the temporo-parietal junction by combining data-driven
parcellation, complementary connectivity analyses, and functional decoding.
\textit{Neuroimage}, 2013.

\bigskip
\textit{Review and opinion papers}

Eickhoff SB, Thirion B, Varoquaux G, \textbf{Bzdok D}.
Connectivity-based parcellation: critique \& implications.
\textit{Human Brain Mapping}, in press.

Eickhoff, SB \& \textbf{Bzdok D}.
Neuroimaging and modeling. Where is the road to clinical application?
\texit{Der Psychiater}, 2014, in press. 

Eickhoff SB \& \textbf{Bzdok D}.
[Statistical meta-analyses in imaging neuroscience.]
\textit{Klinische Neurophysiologie}, 2013, 44:199-203.

\bigskip
\textit{Book chapters}

\linebreak
\textbf{Bzdok D} \& Eickhoff SB.
Statistical learning of the neurobiological of schizophrenia.
In: \textit{The neurobiology of schizophrenia}, Springer, Heidelberg.

\newpage

\section*{1 Introduction}

\subsection*{1.1 Analytical versus heuristic views of nature}

The world around us is complex and volatile.
%
A large proportion of human research efforts are meditated by the 
"the unreasonable effectiveness of mathematics in the natural sciences."
This was phrased by the Hungarian-American
physicist, mathematician, and Nobel laureate Eugene P. Wigner (1960).
The language of mathematics is a powerful tool to
describe, formalize, and predict phenomena in nature.
The author emphasizes that it is not imperative that
natural regularities exist in the world. He goes on to
say that it might be even more surprising that humans can actually
find and use these regularities to their advantage.
%
Starting from simple axiomes we have derived always more complicated
properties of and relationships between mathematical objects by formal proofs
(Connes A., "A view of mathematics").
A logical pyramid of theoreoms is built that lead to always
more general asserations.
We also have detailed knowledge of the limitations of these mathematical
assertions.
%
On the one hand,
an identical regularity can often be equally well described in very distant
branches of mathematics.
On the other hand,
identical mathematical conclusions have also reemerged from derivation of
a priori unrelated assertions.
%
Indeed, the same formal language has proofed very apt in
the study of completely unrelated topics and diverging scientific disciplines.
This includes movements of celestial objects in the universe studied in astronomy,
metabolism pathways governing the inner life of the cell studied in biochemistry,
and
ABC
.
%
Many rules about the world can thus be perfectly grasped.
As another simple example,
Fibonacci numbers (1, 1, 2, 3, 5, 8, 13, etc.)
reappear in many natural phenomena.
The number of petals of a flower and the spirals of a pinapple tend
to be a Fibonacci sequence.
The family tree of honey bees is also governed by Fibonacci regularities.
Even the proportions of human finger bones follow this formalism.
Knowledge of such mathematical regularity 
allows to impose logical structure on the external world.
%
It remains an unresolved question whether we have
\textit{discovered} or \textit{invented} mathematics.
Yet, there is probably no doubt
that mathematical conceptualization
evolves as a feature of human cultural evolution (Tomasello, 2001).
Even the most abstract mathematical concepts can
be exchanged between individuals. Consequently,
they can be easily exchanged between generations and geographical distances.
It is interesting to note that there is usually consensus among mathematicians
about the architecture of their discipline.
In this way, mathematical formalism
appears to be one of the most
powerful tools and most defining properties of the human species
(S. Dehaene, "The Number Sense").
Eugene Wigner concludes the above cited paper with the following words:
"The miracle of the appropriateness of the language of mathematics
for the formulation of the laws of physics is a wonderful gift
which we neither understand nor deserve. We should be grateful for
it and hope that it will remain valid in future research [...]."
(1960, p. 14).


However, this dogma has repeatedly been challenged in the most recent years
(Halevy et al., 2009).

omnipotence of mathematics



classes of phenomena

Every probabilistic model is a superset of a deterministic model (because the deterministic model could be seen as a probabilistic model where the probabilities are restricted to be 0 or 1)

possible paradigm shift

Peter Norvig 
American computer scientist

modern heuristic models can have millions of parameters

Many phenomena in science are stochastic, and the simplest model of them is a probabilistic model; I believe therefore that probabilistic models are our best tool for representing facts about language, for algorithmically processing language, and for understanding how humans process langu


Norvig:
-> insight: simple models fed by large quantities of data perform
better than sophisticated models fed by scarce quantities.
-> Whether a mathematical formalism describes nature well or badly
depends on the quality and quantity of the available data resources.
-> Observing a phenomenon in nature a sufficient number of times
might be sufficient to extract the regularities of its behavior.

Data-driven results can suffer from difficulty to interpret them.

learned staitsical translation models on non-annotated raw
input.
The quality of the data is outweighed by their quantity.
condition probabilities of n-grams.
informal accounts of English grammar count several thousand pages.
statistical machine translation

Machine translation:  of major systems are trained and probabilistic, mostly relying on
probabilistic hidden Markov models.

All current search engines work this way.

The class of hidden markov models (HMM) become a dominanting feature.
- a heuristic, probabilistic mathematical model.

advertisement and politics

rule and human-imposed logic/grammar systems never achieved
convincing success (P. Norvig, 2011, "On Chomsky").
Such heuristic models might indeed better represent the reality of
human communication.

modelling the world

Computational linguistic has turned into a 


Similar switch may be expected in algorithmic trading, self-driving
cars.

humans create machines

But I made the switch: after about 14 years of trying to get language models to work using logical rules, I started to adopt probabilistic approaches


give three examples

- language
- taxics

Protein-Protein Interaktion
SOLL SYMBOLISIESEN
aus Aminosäure-Sequenz allein-> Problem: aus dieser 1D Struktur die 3D Faltung des Proteins abzuleiten/vorherzusagen. 30.000 Protein Strukturen k anderen nicht
momentan v.a. Protein Interaktions-orte identifzieren
Die Ermittlung der natürlichen Proteinstruktur mit physikalischen Methoden ist zwar für viele, aber bei weitem nicht alle, Proteine möglich und zeitlichem Aufwand verbunden.
Pharma-Industrie:
ohne biologisches Wissen
Vorhersagen ist auf mehreren Abstraktionsebenen möglich:
Von „Primärstruktur“ -> Quartärstruktur vorhersagen; Sekundär: Wasserstoff-brücken zwischen benachbarte Teilen von AS; Tertiär: 3D Struktur, genaue Atompositionen darin
Quartär: Protein-Einheit -> Protein-Protein Interaktion vorhersgaen
Wichtige Biologische Strukturen, die ML finden soll: Bindungsstellen, Faltungsart,
„Chemoinformatics"
-> eines der wichtigsten Ziele der Bioinformatik und theoretischen Chemie, diese Eigenschaften von Protein zu entdecken, ohne die Struktur zu kennen - mit AS-Sequenz allein; schneller -> rausfinden, ob ein Protein toxisch oder nicht sein wird: kürzlich deep learning / neuronales Netzwerk -> bessere Vorhersage als alles was in Academia oder Industrie jemals vorher gab im Team hatte Ausbildung in Biologie oder Lebenswissenschaften oder Pharmazie; 2 Wochen investiert


-> dieses andere Beispiele haben uns gezeigt, dass
schlechtes Modell mit vielen Daten ist besser als gutes Modell mit wenig Daten
reletiviert nun eben gerade das Primat der Mathematik. Derjenige macht die besten Übestzne, der die größte Menge und beste Qualitäten an der die besten Annahmen



Their program does not decode the numbers. ReCaptcha works by showing the user two images, only one of which is known to the system. The user the known control and which is the genuinely unknown, so has to solve both. If the control image is solved correctly, it is assumed that the submitted correct too, so the system updates its database with this value
etwas allge
digitalization von alten Büchern (bei denen sich herkömmliche Algos schwertun mögen)
Google has been using reCAPTCHA to digitize content for Google Books


von algorithmischer Modellierung: das Phänomen in der Natur, das wir hier als wenig bekannt komplex und teilweise vielleicht nicht



"Big-data is not interesting, if it's not for ML."
"ML is a set of software algorithms that learn patterns in data without explicit programming model."
"We have to relax our presumptions of control. Because we can guide the process, but we have no precise control over it." "You do not control what you make but you control the process of its creation."
"We can process things that exceed beyond human comprehension - As a more humble approach to the world." "Tradionally prosaic, non-tec businesses will be revolutionized by applying big-data and applying ML."
The Punkt ist - es gibt aktuell vielleicht keine Alternative.

Pietsch:
big-data science requires (i)data representing all relevant configurations of the examined phenomenon. For complex phenomena, this implies high-dimensional data, i.e. data sets in large number of observations covering a wide range of combinations of these parameters.
For many phenomena there exists a relatively sudden change when data-driven approaches become effective (Halevy et al. 2009)—a transition point that could be called adata thresh explanation for its existence: ‘For many tasks, once we have a billion or so examples, we essentially have a closed set that represents (or at least approximates) what we need, withou
On the other hand, the models used in big-data science are often much simpler than the elaborate theoretical structures developed by humans mainly for the purpose of data reduction represent a system, there is no need for complex, hierarchical models anymore—at least with respect topredictions. As Halevy et al write: ‘invariably, simple models and a lot of data on less data.’ (2009, 9) Before providing an overview of the characteristics that distinguish big-data modeling from more conventional scientific modeling let us first take a look at t differences.


For many phenomena there exists a relatively sudden change when data-driven approaches become effective (Halevy et al. 2009)—a transition point that could be called a data threshold. 

Halevy et al give a plausible explanation for its existence: ‘For many tasks, once we have a billion or so examples, we essentially have a closed set that represents (or at least approximates) what we need, without generative rules.’


Pietsch:
The notion of a qualitative change from hierarchical to horizontal modeling is further corroborated by a concurrent paradigm shift in statistics, which has been described as a transition from parametric to non-parametric modeling (e.g. Russell & Norvig 2010, Ch. 18.8), from data to algorithmic models (Breiman 2001), or from model-based to model-free approaches. 

While the parametric model consists in a relatively simple mathematical equation, the non-parametric model consists in all the data and an algorithmic procedure for making predictions.

The difference between the two types of models is striking: While parametric models usually are simple equations, the non-parametric models consist in the original data plus an algorithm to derive predictions from the data.




\subsection*{1.2 Two cultures of statistical modelling}

Statistics is a branch of mathematics that has arguably been the
overall most successful information science.
%
Given its ecclectic character, it may come as no surprise that statistics
has developed both analytical and heuristic strategies
to model regularities of phenomena in nature.
Yet, analytical and heuristic statistical cultures
have emerged independently (Breiman, 2011).
They are diverging with regard to historical origin, mathematical foundation,
and conceptual purpose.

The overwhelming majority of statisticians
follow an analytical regime by
adhering to \textit{classical statistics} (CS) and performing
\textit{data modelling}. They hold that nature can be viewed as a black box
whose inner workings can be described by a small set of
underlying variables.
It is up to the statistician in charge
to choose the model that best reflects nature.
Data is then used to estimate the parameters of that pre-specified model.
Classical statistics have dominated research at the universities
for almost 90 years now.
Well known members of the CS family include Student's t-test, ANOVA,
and Chi-squared test.
\textit{Statistical hypothesis testing} has been introduced in the beginning
of the last century (Fisher, 1925; Neyman and Pearson, 1928).
The same approach is stil practiced today in the same form (Goodman, 1999).
%
The ensuing \textit{p-value} measures how likely the observed data
are assuming the non-preferred null hypothesis to find indirect evidence
for the preferred alternative hypothesis.
%
Despite the prevailing presence of p-values,
it has not been conceived by Fisher as an acid test
to judge existing versus non-existing effects in nature.
Rather, the intention was a preliminary tool to
filter which potential effects should be more explicitly tested (Nuzzo, 2014).
%
Notably, if the assumed model is a bad description of
the natural phenomenon under study
then the drawn conclusions may be wrong.
%
Nevertheless, statistical hypothesis testing probably fit perfectly
in its time of conception and adoption.
It was designed for use with mechanical calculators
(Efron and Tibshirani, 1991). Gaussian distributional assumptions
have been very useful in many instances for
mathematical convenience and, hence, computational tractability.
Additionally, it suited perfectly the Popperian view of
\textbf{critical empirism} in academic discours:
scientific progress is to be made by continuous replacement of current
hypotheses by always more pertinent hypotheses
using verification and falsification (193X).
%
In sum,
classical statistics was mostly fashioned
for small problems with few data points that can be modeled 
by plausible models with a small number of parameters chosen by the
investigator.
 


- origin of 5% significance threshold: harvest (Cowles and Davis, 1982)


In contrast, only a small minority of statisticians
follow a heuristic regime by
adhering to \textit{statistical learning} (SL) and performing
\textit{algorithmic modelling}.
This statistical framework is frequently adopted by computer scientists,
physicists, and ingenieurs that are typically working in industry
rather than academia (Breiman, 2001).
They hold that natural phenomena
can be studied by estimating regularities in the inputs and
outputs to the black box without making assumptions
about the \"true\ mechanisms in its internals.
Please note that SL here summarizes the seemingly more specific terms data-mining, pattern recognition, and machine learning that are often employed inconsistently.
Members of
the SL family include k-means clustering,
Lasso/Ridge regression, and support vector machine classification.


3 components that need to apply when deciding whether a certain problem is a possible target for ML a) a pattern exists (If there is no pattern, then there is no harm in trying ML.)

b) we can not pin down the pattern analytically/mathematically (if we can pin down the problem mathematically GENAU FASSEN K use ML, but it might not create the best model)
c) we have data on the problem



1. es existiert ein Muster

2. wir können das Muster nicht analytisch greifen

3. wir haben Daten zu diesem Problem




In the past fifteen years, the growth in algorith- micmodelingapplicationsandmethodologyhas beenrapid.Ithasoccurredlargelyoutsidestatis- tics in a new community—often called machine learning—thatismostlyyoungcomputerscientists

Mul6variate (bo]om-up): (benutzt von Industriesta6s6kern und jungen ; mit Bereich „künstl

Befreiung von Glocken-förmigen Verteilungsannahmen
-algorithmicmodelling: algorithmische Datenmodellierung behandelt die DateneigenschaYen als unbekannt -> Inhalt der Blackbox ist mysteriös
-Classifica6on is the analogue of regression when the variable being predicted is discrete, rather than con6nuous
-2% aller Sta5s5ker
- data mechanisms unknown
-eine Funk5on erschaffen, die aus x auf welchem Wege auch immeryvorhersagen kann
-Ergebnis: konkreter Prozentsatz, wie oY der vorher gelernte Algorithmus eine unbekanntes/ungesehenesdatensetklassifizieren kann; is simply the frac6on of examples in the test set for which the correct label was predicted -Blackboxzwarunbekannt,aberüberErfoplgundMisserfolgunterschiedlicherKategorienvonClassfifiernkanndieMachnismenderBlackboxin -Model wird vom Algorithmus erzeugt
- studiertes Phänomen hoch komplex und in Teilen vielleicht nicht zu verstehen
-keine subjek6veOpera6onilisierungnö6g
-Nachteil:Vohersageerfolgund Einfachheit/Interpre6erbarkeit im Konflikt -> machen zwar keine Annahmen über die Blackbox, sind abe -nicht unbedingt zwangsläufig besser, aber komplexere Perspek6ve auf diesePhänonemeder Natur
- >100 Daten eigener, 100 und mehre Parameter
"Predictionisrarelyperfect.Thereareusu- allymanyunmeasuredvariableswhoseeffectis referred to as “noise.” But the extent to which the model box emulates nature’s box is a measure the natural phenomenon producing the data."
".Ithasevolvednaturallyinenvironments with lots of data and lots of decisions. But you"

Second the algorithmic modeling culture (subscribed to by 2% of statisticians and many researchers in biology, artificial intelligence, and other fields that deal with complex phenomena), which holds that nature's black box cannot necessarily be described by a simple model. Complex algorithmic approaches (such as support vector machines or boosted decision trees or deep belief networks) are used to estimate the function that maps from input to output variables, but we have no expectation that the form of the function that emerges from this complex algorithm reflects the true underlying nature.

it is that they produce a form that, while accurately modeling reality, is not easily interpretable by humans, and makes no claim to correspond to the generative process used by nature. In other words, algorithmic modeling describes what does happen, but it doesn't answer the question of why.

Instead he asks us to be satisfied with a function that accounts for the observed data well, and generalizes to new, previously unseen data well, but may be expressed in a complex mathematical form that may bear no relation to the "true" function's form (if such a true function even exists).

computers -> since 1980 new wave of computer-based methods -> deemphasize mathematical tractability
(a lot of classical stats is motivated by convenience of Gaussian distribution)

computer-intensive statistical techniques

methatmics less prominent, but justification for validity of algorithms

intricacies between stats approaches (Efron, 2005)

presupposes few modeling assumptions, e.g. allows for a wide range of functional dependencies or of distribution functions.

data representing all relevant configurations of the examined phenomenon.

In non-parametric modeling, predictions are calculated on the basis of all data. There is no detour over a parametric model that summarizes the data in terms of a few parameters. iii) While thisrenders non-parametric modeling quite flexible with the ability to quickly react to unexpected data, it also becomes extremely data- and calculation-intensive. This aspect accounts for the fact that non-parametric modeling is a relatively recent phenomenon in scientific method.

Non-parametric models allow for novel ways to deal with complexity:
Conventional parametric modeling in terms of equations, describing for example functional dependencies or distribution functions, already presupposes that the picture has been reduced to a small number of parameters and to relatively simple functional relationships. By contrast, algorithmic modeling does not have such restrictions. It relies less on sophisticated mathematics and more on a brute-force execution of a large number of steps, when for example an algorithm searches a large data-base for similar cases. 

By contrast, parametric modeling usually emphasizes understanding. While parametric modeling often correlates with a realist and reductionist viewpoint, non-parametric modeling has instrumentalist and pluralist connotations.

Higherpredictiveaccuracyisassociatedwith more reliable information about the underlying data mechanism.



reach conclusions


A statistical model expresses relationships between a set of variables
whose parameters are learned by training data points.



Different philosophies
Fortunately, there are two diverging families of statistical methods. These can be framed as classical statistics (CS) and statistical learning (SL).  In yet a different terminology, CS and SL roughly correspond to parametric and non-parametric statistics, respectively (Nichols and Holmes, 2002). It is also important to appreciate that certain statistical methods cannot be easily categorized by this distinction. Statistical methods span a continuum between the two poles of CS and SL. Nevertheless, the two families of statistical methods can be distinguished by a number of properties.
One of the key differences becomes apparent when thinking of the neurobiological phenomenon under study as a black box (Breiman, 2001). CS typically aims at modeling the black box by making a set of accurate assumptions about its content, e.g. the nature of the signal distribution. SL typically aims at finding any way to model the output of the black box from its input while making the least assumptions possible (Abu-Mostafa et al., 2012). In CS the phenomenon is therefore treated as partly known (i.e., the stochastic processes that generated the data), whereas in SL the phenomenon is treated as complex, completely unknown, and partly unknowable. In this way, CS tends to be analytical (i.e., imposing mathematical rigor on the phenomenon), whereas SL tends to be heuristic (i.e., finding useful approximations to the phenomenon). CS assumes a given statistical model at the beginning of the investigation, whereas in SL the model parameters and at times even the model structure (e.g., in decision-tree based learning algorithms) are generated in the process of the statistical investigation. In more formal terms, CS therefore closely relates to parametric statistics for confirmatory data analysis, whereas SL closely relates to non-parametric statistics for exploratory data analysis. In more practical terms, CS is typically applied to experimental data, where the variables were controlled by the investigator (i.e., system under studied is perturbed), while SL is typically applied to observational data without such structured influence (i.e., system is left unperturbed) (Domingos, 2012). The work unit for CS is the quantified significance associated with a statistical relationship between few variables given a pre-specified model. The work unit for SL is the quantified robustness of patterns between many variables or, more generally, the robustness of special structure in the data (Hastie et al., 2011). CS therefore tests for a particular structure in the data, whereas SL explores and discovers structure in the data. Formally, CS thus resorts to data modeling, e.g., imposing an a priori model in a top-down fashion, whereas SL resorts to algorithmic modeling, e.g. fitting a model as a function of the data at hand in a bottom-up fashion. Intuitively, the truth is believed to be in the model (cf. Wigner, 1960) in a CS framework, while it is believed to be in the data (cf. Halevy et al., 2009) in a SL framework. As a drastically oversimplified, yet educational conclusion, CS preassumes and tests a model for the data, whereas ML learns a model from the data.

Taken together, CS assumes that the data behave according to known mechanisms, whereas SL exploits computer algorithms to treat the data mechanisms as unknown. Again, we caricature the dichotomy in statistical methodology in neuroimaging by oversimplification for pedagogical reasons. Bayesian statistics (Ashburner et al., 1997; Friston et al., 2002) do for example not strictly qualify for only CS or SL. Bayesian statistics are orthogonal to the CS-SL distinction and can be adopted in both methodological families in various flavors. Neither can the terms univariate versus multivariate (i.e., relying on one versus more than one input variable) be clearly grouped into either CS or SL. CS can be divided into uni- and multivariate groups of statistical tests (Ashburner and Kloppel, 2011; Friston et al., 2008), while SL is largely focused on high-dimensional (hence intrinsically multivariate) problems. However, univariate CS methods are routinely incorporated into SL analysis pipelines (e.g., ANOVA [CS] for feature selection [SL] or significance testing [CS] of a classification result [SL]). Importantly, neither CS nor SL can generally be considered superior. This is captured by the no free lunch theorem stating that no single statistical strategy can consistently do better in all circumstances (Domingos, 2012; Wolpert, 1996). The challenge relies in choosing the statistical approach that is best suited to the neurobiological phenomenon under study and the neuroscientific research object at hand.

Different histories
The independent historical origin of the two statistical families is even witnessed by the most basic terminology. In the CS literature inputs to statistical modeling are classically called independent variables or, more recently, predictors, while these are commonly referred to as features in the SL literature (Hastie et al., 2011).


SL:
- In the big-data era the datasets are continously increasing in size and complexity. As a consequence, an increasing number of CS tools become less attractive (Manyika et al., 2011). This is a trend that is not specific to neuroimaging research but also takes places in other scientific disciplines, for example weather forecasting and economic predictions, and beyond (Breiman, 2001)
- users: SL have been used for decades by statisticians in industry (cf. Daniel and Wood, 1971); mostly non-statisticians, especially computer scientists, physicists, and engineers; approximately a minority of 2% according to Breiman2001
- middle of the 80'ies new algorithms neural networks and decision trees
- middle of the 95'ies: SVMs by Vapnik: SVM proofed to be better than the back then de-facto-standard neural networks in both classification and regression problems
- VC bounds by Vapnik -> most relevant theoretical contribution to SL, because it mathematically proofs that learning is possible
- Classification approaches were early used in structural neuroimaging (Herndon et al., 1996) and contributed to improved image preprocessing performance (Ashburner and Friston, 2005).
- Their popularity of SL in neuroimaging increased dramatically in the attempt of "mind-reading" or "decoding" cognitive processes from neural activity patterns (Haynes and Rees, 2005; Kamitani and Tong, 2005)
- Recently increasing effort towards disease classification, although often not in an aim to characterize underlying pathophysiology

Different theories
As a result of their separate historical origin, CS and SL rely on distinct conceptual frameworks, revolving around hypothesis testing versus statistical learning theory.

CS:
In the spirit of Karl Paper's (1935/2005) emphasis on falsification of hypotheses as the foundation of empirical research, CS started in adopting a principled approach based on formulating and testing hypotheses (Fisher, 1925; Neyman and Pearson, 1928). The rationale behind hypothesis falsification is that even a lot of evidence cannot confirm a given theory in an inductive way, while a single counter example is able to proof a theory wrong in a deductive way.

The hypotheses to be tested should be a precisely formulated derivation from new theories extending the current state of a field. The new hypotheses should precisely contradict what is generally believed to be true in that field. The statement reflecting general belief is expressed in the null hypothesis (H0), while the innovative hypothesis is expressed in the alternative hypothesis (H1). The null hypothesis should be directly deducible from the alternative hypothesis alone.

The investigator seeks to proof the null hypothesis wrong thus confirming the initially intended alternative hypothesis. If the null hypothesis is successfully rejected by the test statistic, rendering the alternative hypothesis more likely. The latter hypothesis more confidently explains the phenomenon under study and should hence be accepted as the new general believe.
To exclude the possible rejection due to the idiosyncracy of the tested sample, not true in the general population, although the null hypothesis is correct, a prespecified threshold needs to be exceeded for confident rejection. This is the significance threshold alpha, conventionally 0.05.
That is, if the result has a probability of only 5% of being true under the null hypothesis (P(result|H0)), it is deemed significant. Please note that this statement of significance does not equate with the probability of the null hypothesis (P(H0)), alternative hypothesis (P(H1)), result of the test statistic (P(result)), or the null hypothesis given the result (P(H0|result)) (Markus, 2001; Pollard and Richardson, 1987).
It is important to appreciate that such inferential statistics can neither find a piece of "truth" nor "proof" a certain statement. That is because every rejection of H0 to choose H1 cannot exclude the probability of committing an alpha error. Put differently, statistical hypothesis testing does not lead to truth but to probabilities of how well the result can be reconciled with H0. Importantly, a significant classical test statistic is a direct judgment about the unfavored hypothesis (H0) but not the favored hypothesis (H1). On the one hand, a significant result indirectly supports the hypothesis initially advocated by the investigator (H1). On the other hand, a non-significant result does not confirm H0 but indicates that the test statistic was not suitable to assess the hypothesis pair. In fact, a non-significant result is not an instance of falsification as it does neither proof nor disproof a theory. The most important reason for a non-significant result is an insufficient sample size.

In sum, successful rejection of H0, intended by the investigator, indicates that H1 should temporarily be assumed to be correct (until it is rejected itself in the future), instead of proofing or confirming a certain theory. Taken together, classical hypothesis testing continuously replaces hypotheses explaining a phenomenon in nature for better hypotheses with more empirical support in a Darwinian selection process.

Another relevant property is the dependence of significance of the sample size. Any H0 can be rejected and any difference in the sense of H1 can reach significance given a sufficiently large sample.
Inexact hypotheses of the type "there is a statistical relationship between variable A and B" (such as in a Pearson correlation) can thus be confirmed in a very large sample even if the relationship is very small.

Given the centrality of hypothesis evaluation, CS entails statistical issues when making such a comparison more than once (multiple comparisons problem) that needs to be accounted for. Note that this property of CS has perhaps no obvious analogue in SL.

SL:
In SL two equally important theoretical frameworks peacefully coexist. The concept of Vapnik-Chervonenkis dimensions (VC dimensions) is crucial in statistical learning theory, while the concept of bias-variance tradeoff is highly relevant in both abstract theory and practical everyday application of SL (Hastie et al., 2011). The VC dimensions mathematically formalize the circumstances under which learning can be successful (Vapnik, 1989, 1996). This comprises any instance of learning from a number of observations to derive general rules that capture properties of phenomena in nature, including human learning humans and computer learning. As a side note, both human and computer learning are theoretically more conceivable in a probabilistic rather than deterministic sense (Abu-Mostafa et al., 2012; Dayan et al., 1995; Friston, 2010; Gregory, 1980). VC dimensions provide a probabilistic measure whether a certain model is able to learn a distinction with respect to a given dataset. Formally, the VC dimensions measure the complexity capacity of a class of functions by counting the number of data points that can be cleanly divided (i.e., "shattered") into distinct groups as a result of the flexibility of the function set. Intuitively, the VC dimensions provide a guideline for the largest set of examples fed into a learning function such that we can guarantee zero distinction errors. They can be viewed as the effective number of parameters or the degrees of freedom. In this way, the VC dimensions concretize the circumstances under which a class of functions is able to learn from a finite amount of data to successfully predict a given phenomenon in previously unseen data. Practically, good models have finite VC dimensions: Its assessed performance in a sufficiently large amount of data used for learning confidently approximates the theoretically expected performance in any new data. Bad models have infinite VC dimensions: Regardless of the available amount of data, it is impossible to make generalization conclusions on unseen data. Note that the VC dimensions are unrelated to the target function (i.e., unknown mechanisms underlying the studied phenomenon in nature) but relate to the models used to approximate that target function. Even if the VC dimensions predict learning as very likely, it is still possible that generalization from a concrete dataset fails. Yet, such a dataset is theoretically unlikely to occur. Although the concept of VC dimensions is the most important achievement in statistical learning theory (Abu-Mostafa et al., 2012), they can only be explicitly computed for simple models. Hence, investigators are often restricted to an approximate bound for VC dimensions limiting their use to theoretical considerations.
The bias-variance tradeoff is a stand-alone theoretical framework that is more applicable in practice and provides a different angle on learning from data (e.g., Geman et al., 1992). It is more explicitly mentioned in SL, although it applies to statistical analysis in general. As George Box famously noted , a model is an oversimplication of the studied phenomenon, and thus a distortion of the truth. Choosing the right model amounts to the right balance between bias and variance. Bias denotes the difference between the target function and the average of the function space derived from a model. Intuitively, the bias describes how rigid the functions derived from a model are. When using high-bias models, even for larger perturbations of the data statistical learning results in virtually the same best approximating function. Variance denotes the difference between the best approximating function from the function space and the average of the function space. Intuitively, variance describes how different the functions derived from a same model can be. When using high-variance models, even for small perturbations of the data statistical learning results in very different best approximating functions. In sum, as the model complexity increases, bias tends to decrease and variance tends to increase, and vice versa. A simple model instantiates simple functions featuring high bias and low variance. They do not well simulate the target function (i.e., bad approximation) but are good at generalizing to unseen data (i.e., good generalization). Contrarily, complex models feature low bias and high variance. They have a better chance of well approximating the target function at the price of being less able to generalize well to new data. That is because complex models tend to adapt too well to the data used for learning to the point of fitting noise. This is called overfitting and occurs when the learning model integrates information independent of the target function and idiosyncractic to the dataset. The bias-variance decomposition captures the fundamental tradeoff in statistical modeling between approximating the behavior of the studied phenomenon and generalizing to new data describing that behavior. If we know the target function, we can explicitly compute the bias-variance tradeoff, in contrast to the inability of computing the VC dimensions of non-trivial models. The bias-variance tradeoff illustrates the core challenge in SL of choosing a model that is able to find a well approximating function given the available data. The bias-variance tradeoff can also practically explain why successful application of SL mainly relies on three factors: i) the (known) amount of available data, ii) the (not precisely known) amount of noise in these, and iii) the (unknown) complexity of the target function.








\subsection*{1.3 The human brain as a complex phenomenon}

Opportunities lie for example in medicine

Bild von einem meiner Kollegen aus Paris entwendet
trotz 200 Jahren Hirnforschungs immer noch keine universelle Hirntheorie zu einigen, die auf von Hirnforschern unterschiedlicher gebiete al würde;
-> hat den unangenehmen Effekt, dass besonders in neurowiss. Systembiologie sehr schwer genau Hypothesen zu entwerfen, di eman dann gezielt experimentell testen und so
Sich den Mechanismen von neuronalener Verarbeitung mit einem Minimum an nötigen Vornahmen, AUS DATEN lernen zu nähern, ist daher ein besonderen attraktiver Zugang


It has been argued that classical functional neuroimaging applications
are insufficient to infer structure-function relationships
without formal modelling (Stephan, 2004).
-> towards formalized descriptions of SFRs


Biology has a small legacy of abstract theorizing.
The present thesis is an attempt to the study the brain using
heuristic, data-driven approaches.

neurobiological mechanisms

a intrinsically probabilistic process
-> perception depends on the phase of intrinsic oscillations cycles

unresolved nature-nurture debate

Given the lack of focal structural changes in almost all psychiatric dis- eases

consciousness




epistemology/brain

why ML is attractive
nature produces data in a black box whose insides are complex, mysterious, and,atleast,partlyunknowable.Whatisobserved


\subsection*{1.4 the curse of dimensionality????}

Richard Bellman

Inanastronomyandstatisticsworkshopthis year, a speaker remarked that in twenty-five years we have gone from being a small sample-size science toaverylargesample-sizescience.Astronomical data bases now contain data on two billion objects comprisingover100terabytesandtherateofnew information is accelerating.

HCP / TERA


- difficulty of finding pertinent hypotheses and increasing motivation towards multi-site studies and ever increasing sample sizes currently challenge the CS foundation of neuroimaging




\subsection*{1.5 Imaging neuroscience}






\subsection*{1.6 Statistical learning approaches in brain imaging}









Typical big-data problems concern classification or regression of an output variable y with respect to a large number of input parameters x, also called predictor variables or covariates, on the basis of large training sets. 


Note that this curse of dimensionality does not automatically apply to all big-data algorithms. To the contrary, it occasionally turns out helpful to artificially increase the dimensionality of the parameter space in methods like decision trees or support vector machines 





\section*{2 Unsupervised modelling of brain regions}

\subsection*{2.1 Motivation}
\subsection*{2.2 Methodological approach}
\subsection*{2.3 Experimental results}
\subsection*{2.4 Discussion}



\section*{3 Supervised modelling of brain networks}

\subsection*{3.1 Motivation}
\subsection*{3.2 Methodological approach}
\subsection*{3.3 Experimental results}
\subsection*{3.4 Discussion}


\section*{4 Semi-supervised modelling for structure discovery
and structure inference}

\subsection*{4.1 Motivation}
\subsection*{4.2 Methodological approach}
\subsection*{4.3 Experimental results}
\subsection*{4.4 Discussion}




\section*{5 Conclusion}






\bigskip
\section*{6 References}

\bibliographystyle{plainnat}
\bibliography{danilos_endnote_list}

\end{document}
